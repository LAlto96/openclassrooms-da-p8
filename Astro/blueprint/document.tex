\documentclass{article}
% Allow the usage of UTF-8 characters
\usepackage[utf8]{inputenc}
% Allow the usage of graphics (.png, .jpg)
\usepackage{graphicx}
\usepackage{array,multirow,makecell}
\setcellgapes{1pt}
\makegapedcells
\usepackage{cellspace}
\cellspacetoplimit 10pt
\cellspacebottomlimit 10pt
\newcolumntype{R}[1]{>{\raggedleft\arraybackslash }b{#1}}
\newcolumntype{L}[1]{>{\raggedright\arraybackslash }b{#1}}
\newcolumntype{C}[1]{>{\centering\arraybackslash }b{#1}}
% Start the document
\begin{document}

% Create a new 1st level heading
\section{Projet}
\subsection{Vision Stratégique}
Fournir un outil permettant de synthétiser et faciliter l'accès à la connaissance pour un astronome amateur. Ainsi un dashboard sera crée afin de donner des outils permettant à l'utilisateur de préparer son observation du ciel. Beaucoup d'Open Data existe dans le domaine, la majorité des données provenant d'organisme publiques tels que la \textit{NASA}) 
\subsection{Cas d'usages}
\begin{tabular}{|S{L{5cm}}|S{L{5cm}}|}
\hline Besoin & Réponse au besoin  \\
\hline Une zone d'observation sans pollution lumineuse & Une carte référencent les enregistrements de pollutions lumineuse\\
\hline Un ciel sans Lune & Indicateur en fonction du calendrier Lunaire \\
\hline Quels sont les objets observables dans le ciel ? & Affichage des objets sur un planisphère du ciel (scatter plot pour les objets lointains, line plot pour les objets proches). Ceux-ci pourront êtres filtrés par type d'objet, période d'observation, etc. \\
\hline 
\end{tabular}

% Uncomment the following two lines if you want to have a bibliography
%\bibliographystyle{alpha}
%\bibliography{document}

\end{document}
