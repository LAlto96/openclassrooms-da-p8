\documentclass{article}
% Allow the usage of UTF-8 characters

\usepackage{geometry}
\geometry{hmargin=2.5cm,vmargin=2.5cm}

\usepackage[utf8]{inputenc}
% Allow the usage of graphics (.png, .jpg)
\usepackage{graphicx}
\usepackage{hyperref}
\usepackage{array,multirow,makecell}
% \setcellgapes{1pt}
% \makegapedcells
\usepackage{cellspace}
\cellspacetoplimit 10pt
\cellspacebottomlimit 10pt
\newcolumntype{R}[1]{>{\raggedleft\arraybackslash }b{#1}}
\newcolumntype{L}[1]{>{\raggedright\arraybackslash }m{#1}}
\newcolumntype{C}[1]{>{\centering\arraybackslash }b{#1}}
% Start the document
\begin{document}

% Create a new 1st level heading
\section{Projet}
\subsection{Objectif}
\noindent
Fournir un outil permettant de synthétiser et faciliter l'accès à la connaissance pour un astronome amateur. Ainsi un dashboard sera crée afin de donner des outils permettant à l'utilisateur de préparer son observation du ciel. Beaucoup d'Open Data existe dans le domaine, la majorité des données provenant d'organisme publiques tels que la \textit{NASA}).\\\\
\noindent
L'outil offrira une vue synthétique de différents paramètres apportant une aide à la décision (Pollution lumineuse (naturelle ou artificielle), météo, objets remarquables visibles, position des objets dans le ciel, intensité lumineuse/Albedo. Différents filtres permettront de n'afficher que les informations importantes à l'utilisateur. L'astronomie regorgeant de données, la visualisation devra aller à l'essentiel pour ne pas noyer l'utilisateur.


\subsection{Sources de Données}

- \textit{NASA} : \url{https://data.nasa.gov/}\\
- \textit{Datastro.eu} : \url{https://datastro.eu/}\\
- \textit{Wikipedia (IAU Designated constellations)} : \url{https://en.wikipedia.org/wiki/IAU_designated_constellations/}\\
\end{itemize}


\subsection{Cas d'usages/ Blueprint}
\begin{tabular}{|L{3cm}|L{4cm}|L{5cm}|L{3cm}|}
% Zone Observation
\hline Besoin utilisateur 
& Mesures spécifiques à utiliser 
& Visualisation 
& Vue  \\

\hline Une zone d'observation sans pollution lumineuse 
& Une carte référencent les enregistrements de pollutions lumineuse 
& Carte de France 
& Qualité du ciel\\

\hline La météo prévue pour le jour retenu pour l'observation
& \begin{itemize}
\item Couverture Nuageuse
\item Transparence
\item Température
\end{itemize}
& Valeurs des mesures et scoring
& Qualité du ciel\\

\hline Un ciel sans Lune 
& Indicateur en fonction du calendrier Lunaire 
& Scoring en fonction des différentes phases de la Lune
& Qualité du ciel\\

\hline Quels sont les objets observables dans le ciel ? 
& Affichage des objets sur un planisphère du ciel Ceux-ci pourront êtres filtrés par type d'objet, période d'observation, etc. 
& \begin{itemize}
\item scatter plot pour les objets lointains 
\item line plot pour les objets proches). 
\item Les objets seront colorés en fonction de leur type (planètes, étoiles, galaxies, etc.).
\item Bar Plot et Box Plot sur la magnitude des objets en fonction de leur type
\item un filtre en fonction du materiel disponible (télescope, lunette, etc.)
\end{itemize}
& \begin{itemize} 
\item Messiers 
\item Ephemerides 
\end{itemize} \\

\hline Planisphère des constellations
& Declinaison et ascension droite des constellations
& \begin{itemize}
\item Scatter Plot
\item Lien vers la constellation IAU
\end{itemize}
& Carte des constellations\\

\hline 
\end{tabular}


% Uncomment the following two lines if you want to have a bibliography
%\bibliographystyle{alpha}
%\bibliography{document}

\end{document}
